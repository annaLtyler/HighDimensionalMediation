## Discussion

Here we investigated the relative contributions of local and distal gene 
regulation in four tissues to heritable variation in traits related to 
metabolic disease in genetically diverse mice. We found that 
distal heritability was positively correlated with trait relatedness, 
whereas high heritability was negatively correlated with trait 
relatedness. We used a novel high-dimensional mediation analysis 
(HDMA) to identify tissue-specific composite transcripts that are predicted to 
mediate the effect of genetic background on metabolic traits. Transcripts 
in the adipose tissue had the largest loadings in the composite transcript 
suggesting that transcription in adipose tissue had the greatest heritable 
contribution to the response to a high-fat, high-sugar diet. Adipose 
transcripts with high positively loadings were enriched with biological 
processes associated with inflammation. Transcripts in adipose tissue 
with large negative loadings were enriched for mitochondrial functions, 
particularly thermogenesis. The composite transcript robustly predicted 
body weight in an independent cohort of diverse mice with disparate 
population structure. However, gene expression imputed from local 
genotype failed to predict body weight in the second population. Taken
together, these results highlight the complexity of gene expression 
regulation in relation to trait heritability and suggest that heritable 
trait variation is mediated primarily through distal gene regulation.
We speculate that this study demonstrates a general feature of the
heritability of complex traits and diseases and could have significant
consequences for the development of disease treatments.

## Supplemental Discussion

Genetics is indispensable for the dissection of disease mechanisms 
because it is one of the only data modalities that supports causal 
inferences about molecules and disease outcomes \cite{pmid38632401, 
pmid27855690}. It has frequently been assumed that gene regulation in 
\textit{cis} is the primary driver of genetically associated 
trait variation, but attempts to use local gene regulation 
to explain phenotypic variation have had limited success 
\cite{pmid32912663, pmid36515579}. In recent years, evidence
has mounted that distal gene regulation may be an important
mediator of trait heritability \cite{pmid32424349, 
pmid37857933, pmid31051098}. It has been observed
that transcripts with high local heritability explain less
expression-mediated disease heritability than those
with low local heritability \cite{pmid32424349}. Consistent with
this observation, genes located near GWAS hits tend to be 
complexly regulated \cite{pmid37857933}. They also tend to be
enriched with functional annotations, in contrast to genes with
simple local regulation, which tend to be depleted of functional 
annotations suggesting they are less likely to be directly involved 
in disease traits \cite{pmid37857933}. These observations are 
consistent with principles of robustness in complex systems 
in which simple regulation of important elements leads to 
fragility of the system \cite{pmid29782925, pmid12082173, pmid27304973}.
Our results are consistent, instead, with a more complex picture 
where genes whose expression can drive trait variation are buffered 
from local genetic variation but are extensively influenced indirectly 
by genetic variation in the regulatory networks converging on those genes.

Recently, the omnigenic model of complex traits has been proposed, 
which posits that complex traits are massively polygenic and that 
their heritability is spread out across the genome \cite{pmid28622505}. 
In the omnigenic model, genes are classified either as "core genes," 
which directly impinge on the trait, or "peripheral genes," which are 
not directly trait-related, but influence core genes through the complex 
gene regulatory network. HDMA explicitly models a central proposal of 
the omnigenic model which posits that once the expression of the 
core genes (i.e. trait-mediating genes) is accounted for, there 
should be no residual correlation between the genome and the phenome. 
Here, when the composite transcript was taken into account there was 
no residual correlation between the composite genome and composite 
phenome (Fig. \ref{fig:workflow}A). 

Thus, the transcript loadings can be interpreted as indicating higher 
"core-ness" of a transcript. Unlike in the omnigenic model, we did not
observe a clear demarcation between the core and peripheral genes in 
loading magnitude, but we do not necessarily expect a clear separation 
given the complexity of gene regulation and the genotype-phenotype map 
\cite{pmid29906445}.

```{r, echo = FALSE}
#in the paragraph before a supplemental figure is referenced,
#find its number. The figure counter can be observed at the 
#end to verify the order that the figures should be put in,
#and to see if any figures have not been referenced in the 
#test
fig.label <- "fig:Nucb2_eqtl"
fig.counter <- ref_fig(fig.counter, fig.label)
nucb2.eqtl.label <- fig.counter[fig.label]
```

An extension of the omnigenic model proposed that most heritability
of complex traits is driven by weak distal eQTLs that are potentially
below the detection threshold in studies with feasible sample sizes 
\cite{pmid31051098}. This is consistent with what we observed here. 
The transcripts with the largest loadings were strongly distally 
regulated and only weakly locally regulated, suggesting that 
distal gene regulation plays a primary role in driving heritable 
trait variation. We saw further that the patterns of distal 
heritability were not localized to detectable distal eQTL, but 
rather were complex and spread across the genome, even for 
transcripts whose expression was strongly regulated by distal 
factors. For example, \textit{Nucb2}, had a high loading in islets 
and was also strongly distally regulated (66\% distal heritability) (Fig. 
\ref{fig:loading_heritability}). This gene is expressed in pancreatic 
$\beta$ cells and is involved in insulin and glucagon release 
\cite{pmid22108805, pmid23537085, pmid24993278}. Although its 
transcription was highly heritable in islets, that regulation was 
distributed across the genome, with no clear distal eQTL 
(Supp. Fig. \ref{fig:Nucb2_eqtl}). Thus, although distal regulation 
of some genes may be strong, this regulation is likely to be highly 
complex and not easily localized. 

We stress that HDMA is a method for causal hypothesis generation. As 
with any causal inference approach, the output of HDMA can only be 
said to be consistent with causal mediation but does not prove it. 
Proving causality requires experimentation with direct control over 
the mediating variable \cite{pearl2009causality}. The issue of 
experimentation, however, is subtle. The dimension-reduction in HDMA 
is distinguished by the fact that the putative causal intermediates 
can be emergent states defined by the expression of thousands of genes. 
This is a strength, because the mediating variable can be a higher-order 
process such as “macrophage activation and infiltration”, but, in contrast 
to univariate hypotheses at the level of individual transcripts, the 
relevant validation experiment may be technologically infeasible, 
unknowable a priori, or both. Nevertheless, downstream analyses 
of the composite transcripts strongly supports a causal interpretation. 
Indeed, the composite transcripts identified by HDMA are richly 
interpretable in both tissue- and gene-specific manners. The transcripts 
with the strongest loadings were enriched in biological functions 
previously known to be involved in the pathogenesis of metabolic disease, 
such as inflammation in adipose tissue. That these processes were 
identified in this analysis suggests additionally that they have a 
heritable component, and that some individuals are genetically susceptible 
to greater adipose inflammation on a HFHS diet.

Individual high-loading transcripts also demonstrated biologically 
interpretable, tissue-specific patterns. We highlighted
\textit{Pparg}, which is known to be protective in adipose 
tissue \cite{pmid17389767} where it was negatively loaded, 
and harmful in the liver \cite{pmid12805374, pmid12618528, 
pmid16357043, pmid15644454, pmid16403437}, where it was 
positively loaded. Such granular patterns may be useful
in generating hypotheses for further testing, and 
prioritizing genes as therapeutic targets. The 
tissue-specific nature of the loadings also may provide
clues to tissue-specific effects, or side effects, 
of targeting particular genes system-wide.

In addition to identifying individual transcripts of 
interest, the composite transcripts can be used as 
weighted vectors in multiple types of analysis, 
such as drug prioritization using gene set enrichment 
analysis (GSEA) and the CMAP database. In particular,
the CMAP analysis identified drugs which have been 
demonstrated to reverse insulin resistance and other 
aspects of metabolic disease. This finding supports the 
causal role of these full gene signatures in pathogenesis 
of metabolic disease and thus their utility in prioritizing 
drugs and gene targets as therapeutics.

Another useful application of the composite transcripts is
to pair them with cell-type specific genes to generate causal 
hypotheses about changes in cell composition in individual tissues. 
Combining the multi-tissue, transcriptome-wide weighted vectors 
with public databases and data sets thus provides a path for 
generating a wide range of testable hypotheses. Moreover, each 
publically available data set we used for interpretation of the
HDMA results was derived from human tissues or cell lines, thus 
demonstrating the translatability of the HDMA results to humans. 
That the mouse-derived adipose composite transcript was able to 
classify human adipose gene expression in terms of obesity 
and diabetes status further supports the direct translatablility 
of these findings, the utility of HDMA, and the continued 
importance of mouse models of human disease in which it is 
possible to obtain complete transriptomes in mutliple tissues 
across large numbers of individuals. 
 
Altogether, our results have shown that both tissue specificity and 
distal gene regulation are critically important to understanding 
the genetic architecture of complex traits. We identified important 
genes and gene signatures that were heritable, plausibly causal of 
disease, and translatable to other mouse populations and to humans. 
Finally, we have shown that by directly acknowledging the complexity 
of both gene regulation and the genotype-to-phenotype map, we can gain 
a new perspective on disease pathogenesis and develop actionable 
hypotheses about pathogenic mechanisms and potential treatments. 