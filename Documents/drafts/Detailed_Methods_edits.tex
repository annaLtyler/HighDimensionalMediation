% Options for packages loaded elsewhere
%DIF LATEXDIFF DIFFERENCE FILE
%DIF DEL Detailed_Methods_JMM.tex           Mon Jan 27 14:36:58 2025
%DIF ADD Detailed_Methods_post_review.tex   Wed Feb 19 14:52:26 2025
\PassOptionsToPackage{unicode}{hyperref}
\PassOptionsToPackage{hyphens}{url}
%
\documentclass[
]{article}
\usepackage{amsmath,amssymb}
\usepackage{iftex}
\ifPDFTeX
  \usepackage[T1]{fontenc}
  \usepackage[utf8]{inputenc}
  \usepackage{textcomp} % provide euro and other symbols
\else % if luatex or xetex
  \usepackage{unicode-math} % this also loads fontspec
  \defaultfontfeatures{Scale=MatchLowercase}
  \defaultfontfeatures[\rmfamily]{Ligatures=TeX,Scale=1}
\fi
\usepackage{lmodern}
\ifPDFTeX\else
  % xetex/luatex font selection
\fi
% Use upquote if available, for straight quotes in verbatim environments
\IfFileExists{upquote.sty}{\usepackage{upquote}}{}
\IfFileExists{microtype.sty}{% use microtype if available
  \usepackage[]{microtype}
  \UseMicrotypeSet[protrusion]{basicmath} % disable protrusion for tt fonts
}{}
\makeatletter
\@ifundefined{KOMAClassName}{% if non-KOMA class
  \IfFileExists{parskip.sty}{%
    \usepackage{parskip}
  }{% else
    \setlength{\parindent}{0pt}
    \setlength{\parskip}{6pt plus 2pt minus 1pt}}
}{% if KOMA class
  \KOMAoptions{parskip=half}}
\makeatother
\usepackage{xcolor}
\usepackage[margin=1in]{geometry}
\usepackage{graphicx}
\makeatletter
\def\maxwidth{\ifdim\Gin@nat@width>\linewidth\linewidth\else\Gin@nat@width\fi}
\def\maxheight{\ifdim\Gin@nat@height>\textheight\textheight\else\Gin@nat@height\fi}
\makeatother
% Scale images if necessary, so that they will not overflow the page
% margins by default, and it is still possible to overwrite the defaults
% using explicit options in \includegraphics[width, height, ...]{}
\setkeys{Gin}{width=\maxwidth,height=\maxheight,keepaspectratio}
% Set default figure placement to htbp
\makeatletter
\def\fps@figure{htbp}
\makeatother
\setlength{\emergencystretch}{3em} % prevent overfull lines
\providecommand{\tightlist}{%
  \setlength{\itemsep}{0pt}\setlength{\parskip}{0pt}}
\setcounter{secnumdepth}{-\maxdimen} % remove section numbering

\newcommand{\beginsupplement}{%
        \setcounter{table}{0}
        \renewcommand{\thetable}{S\arabic{table}}%
        \setcounter{figure}{0}
        \renewcommand{\thefigure}{S\arabic{figure}}%
     }

\usepackage{lineno}
\linenumbers

\usepackage[compress, super]{natbib}

\usepackage{setspace} \doublespacing

\usepackage{siunitx}

\usepackage[utf8]{inputenc}
\usepackage{amsmath}
\usepackage{algorithm}
\usepackage{algpseudocode}

\algrenewcommand\algorithmicrequire{\textbf{Input:}}
\algrenewcommand\algorithmicensure{\textbf{Output:}}
\ifLuaTeX
  \usepackage{selnolig}  % disable illegal ligatures
\fi
\IfFileExists{bookmark.sty}{\usepackage{bookmark}}{\usepackage{hyperref}}
\IfFileExists{xurl.sty}{\usepackage{xurl}}{} % add URL line breaks if available
\urlstyle{same}
\hypersetup{
  pdftitle={Online methods for: Transcripts with high distal heritability mediate genetic effects on complex metabolic traits},
  hidelinks,
  pdfcreator={LaTeX via pandoc}}

\title{Online methods for: Transcripts with high distal heritability
mediate genetic effects on complex metabolic traits}
\author{}
\date{\vspace{-2.5em}}
%DIF PREAMBLE EXTENSION ADDED BY LATEXDIFF
%DIF UNDERLINE PREAMBLE %DIF PREAMBLE
\RequirePackage[normalem]{ulem} %DIF PREAMBLE
\RequirePackage{color}\definecolor{RED}{rgb}{1,0,0}\definecolor{BLUE}{rgb}{0,0,1} %DIF PREAMBLE
\providecommand{\DIFaddtex}[1]{{\protect\color{blue}\uwave{#1}}} %DIF PREAMBLE
\providecommand{\DIFdeltex}[1]{{\protect\color{red}\sout{#1}}}                      %DIF PREAMBLE
%DIF SAFE PREAMBLE %DIF PREAMBLE
\providecommand{\DIFaddbegin}{} %DIF PREAMBLE
\providecommand{\DIFaddend}{} %DIF PREAMBLE
\providecommand{\DIFdelbegin}{} %DIF PREAMBLE
\providecommand{\DIFdelend}{} %DIF PREAMBLE
\providecommand{\DIFmodbegin}{} %DIF PREAMBLE
\providecommand{\DIFmodend}{} %DIF PREAMBLE
%DIF FLOATSAFE PREAMBLE %DIF PREAMBLE
\providecommand{\DIFaddFL}[1]{\DIFadd{#1}} %DIF PREAMBLE
\providecommand{\DIFdelFL}[1]{\DIFdel{#1}} %DIF PREAMBLE
\providecommand{\DIFaddbeginFL}{} %DIF PREAMBLE
\providecommand{\DIFaddendFL}{} %DIF PREAMBLE
\providecommand{\DIFdelbeginFL}{} %DIF PREAMBLE
\providecommand{\DIFdelendFL}{} %DIF PREAMBLE
%DIF HYPERREF PREAMBLE %DIF PREAMBLE
\providecommand{\DIFadd}[1]{\texorpdfstring{\DIFaddtex{#1}}{#1}} %DIF PREAMBLE
\providecommand{\DIFdel}[1]{\texorpdfstring{\DIFdeltex{#1}}{}} %DIF PREAMBLE
\newcommand{\DIFscaledelfig}{0.5}
%DIF HIGHLIGHTGRAPHICS PREAMBLE %DIF PREAMBLE
\RequirePackage{settobox} %DIF PREAMBLE
\RequirePackage{letltxmacro} %DIF PREAMBLE
\newsavebox{\DIFdelgraphicsbox} %DIF PREAMBLE
\newlength{\DIFdelgraphicswidth} %DIF PREAMBLE
\newlength{\DIFdelgraphicsheight} %DIF PREAMBLE
% store original definition of \includegraphics %DIF PREAMBLE
\LetLtxMacro{\DIFOincludegraphics}{\includegraphics} %DIF PREAMBLE
\newcommand{\DIFaddincludegraphics}[2][]{{\color{blue}\fbox{\DIFOincludegraphics[#1]{#2}}}} %DIF PREAMBLE
\newcommand{\DIFdelincludegraphics}[2][]{% %DIF PREAMBLE
\sbox{\DIFdelgraphicsbox}{\DIFOincludegraphics[#1]{#2}}% %DIF PREAMBLE
\settoboxwidth{\DIFdelgraphicswidth}{\DIFdelgraphicsbox} %DIF PREAMBLE
\settoboxtotalheight{\DIFdelgraphicsheight}{\DIFdelgraphicsbox} %DIF PREAMBLE
\scalebox{\DIFscaledelfig}{% %DIF PREAMBLE
\parbox[b]{\DIFdelgraphicswidth}{\usebox{\DIFdelgraphicsbox}\\[-\baselineskip] \rule{\DIFdelgraphicswidth}{0em}}\llap{\resizebox{\DIFdelgraphicswidth}{\DIFdelgraphicsheight}{% %DIF PREAMBLE
\setlength{\unitlength}{\DIFdelgraphicswidth}% %DIF PREAMBLE
\begin{picture}(1,1)% %DIF PREAMBLE
\thicklines\linethickness{2pt} %DIF PREAMBLE
{\color[rgb]{1,0,0}\put(0,0){\framebox(1,1){}}}% %DIF PREAMBLE
{\color[rgb]{1,0,0}\put(0,0){\line( 1,1){1}}}% %DIF PREAMBLE
{\color[rgb]{1,0,0}\put(0,1){\line(1,-1){1}}}% %DIF PREAMBLE
\end{picture}% %DIF PREAMBLE
}\hspace*{3pt}}} %DIF PREAMBLE
} %DIF PREAMBLE
\LetLtxMacro{\DIFOaddbegin}{\DIFaddbegin} %DIF PREAMBLE
\LetLtxMacro{\DIFOaddend}{\DIFaddend} %DIF PREAMBLE
\LetLtxMacro{\DIFOdelbegin}{\DIFdelbegin} %DIF PREAMBLE
\LetLtxMacro{\DIFOdelend}{\DIFdelend} %DIF PREAMBLE
\DeclareRobustCommand{\DIFaddbegin}{\DIFOaddbegin \let\includegraphics\DIFaddincludegraphics} %DIF PREAMBLE
\DeclareRobustCommand{\DIFaddend}{\DIFOaddend \let\includegraphics\DIFOincludegraphics} %DIF PREAMBLE
\DeclareRobustCommand{\DIFdelbegin}{\DIFOdelbegin \let\includegraphics\DIFdelincludegraphics} %DIF PREAMBLE
\DeclareRobustCommand{\DIFdelend}{\DIFOaddend \let\includegraphics\DIFOincludegraphics} %DIF PREAMBLE
\LetLtxMacro{\DIFOaddbeginFL}{\DIFaddbeginFL} %DIF PREAMBLE
\LetLtxMacro{\DIFOaddendFL}{\DIFaddendFL} %DIF PREAMBLE
\LetLtxMacro{\DIFOdelbeginFL}{\DIFdelbeginFL} %DIF PREAMBLE
\LetLtxMacro{\DIFOdelendFL}{\DIFdelendFL} %DIF PREAMBLE
\DeclareRobustCommand{\DIFaddbeginFL}{\DIFOaddbeginFL \let\includegraphics\DIFaddincludegraphics} %DIF PREAMBLE
\DeclareRobustCommand{\DIFaddendFL}{\DIFOaddendFL \let\includegraphics\DIFOincludegraphics} %DIF PREAMBLE
\DeclareRobustCommand{\DIFdelbeginFL}{\DIFOdelbeginFL \let\includegraphics\DIFdelincludegraphics} %DIF PREAMBLE
\DeclareRobustCommand{\DIFdelendFL}{\DIFOaddendFL \let\includegraphics\DIFOincludegraphics} %DIF PREAMBLE
%DIF COLORLISTINGS PREAMBLE %DIF PREAMBLE
\RequirePackage{listings} %DIF PREAMBLE
\RequirePackage{color} %DIF PREAMBLE
\lstdefinelanguage{DIFcode}{ %DIF PREAMBLE
%DIF DIFCODE_UNDERLINE %DIF PREAMBLE
  moredelim=[il][\color{red}\sout]{\%DIF\ <\ }, %DIF PREAMBLE
  moredelim=[il][\color{blue}\uwave]{\%DIF\ >\ } %DIF PREAMBLE
} %DIF PREAMBLE
\lstdefinestyle{DIFverbatimstyle}{ %DIF PREAMBLE
	language=DIFcode, %DIF PREAMBLE
	basicstyle=\ttfamily, %DIF PREAMBLE
	columns=fullflexible, %DIF PREAMBLE
	keepspaces=true %DIF PREAMBLE
} %DIF PREAMBLE
\lstnewenvironment{DIFverbatim}{\lstset{style=DIFverbatimstyle}}{} %DIF PREAMBLE
\lstnewenvironment{DIFverbatim*}{\lstset{style=DIFverbatimstyle,showspaces=true}}{} %DIF PREAMBLE
%DIF END PREAMBLE EXTENSION ADDED BY LATEXDIFF

\begin{document}
\maketitle

\subsection{Diversity Outbred Mice}\label{diversity-outbred-mice}

Mice were maintained and treated in accordance with the guidelines
approved by the Department of Biochemistry animal vivarium at the
University of Wisconsin. Animal husbandry and in vivo phenotyping
methods were previously published \cite{pmid31343992, pmid29567659}.

A population of 500 diversity outbred mice (split evenly between male
and female) from generates 18, 19, and 21, was placed on a high-fat
(44.6\% kcal fat), high-sugar (34\% carbohydrate), adequate protein
(17.3 \% protein) diet from Envigo Teklad (catalog number TD.08811)
starting at four weeks of age as described previously
\cite{pmid29567659}. Individuals were assessed longitudinally for
multiple metabolic measures including fasting glucose levels, glucose
tolerance, insulin levels, body weight, and blood lipid levels.

\DIFaddbegin \DIFadd{When mice were harvested at 22 weeks of age, their pancreatic islets
were isolated by hand. Insulin per islet was measured, and whole
pancreas insulin content was calculated from the insulin per islet
measure and the total numer of islets per pancreas \mbox{%DIFAUXCMD
\cite{pmid29567659}}\hskip0pt%DIFAUXCMD
.
RNA was isolated from the whole islets and sent to The Jackson
Laboratory for high-throughput sequencing \mbox{%DIFAUXCMD
\cite{pmid29567659}}\hskip0pt%DIFAUXCMD
.
}

\DIFaddend \subsection{Trait measurements}\label{trait-measurements}

Trait measurements were described previously in \cite{pmid29567659}.
Briefly, body weight was measured every two weeks, and 4-hour fasting
plasma samples were collected to measure insulin, glucose, and
triglycerides (TG). At around 18 weeks of age, an oral glucose tolerance
test (oGTT) was conducted on 4-hour fasted mice to assess changes in
plasma insulin and glucose. Glucose (2 g/kg) was given via oral gavage.
Blood samples were taken from a retro-orbital bleed before glucose
administration, and at 5, 15, 30, 60, and 120 minutes afterward. The
area under the curve (AUC) was calculated for glucose and insulin.
Glucose was measured using the glucose oxidase method, and insulin was
measured by radioimmunoassay.

HOMA-IR and HOMA-B, which are homeostatic model assessments of insulin
resistance (IR) and pancreatic islet function (B), were calculated using
fasting plasma glucose and insulin values at the start of the oGTT.
HOMA-IR = (glucose \(\times\) insulin) / 405 and HOMA-B = (360
\(\times\) insulin) / (glucose - 63). Plasma glucose and insulin units
are mg/dL and mU/L, respectively.

\subsection{Genotyping}\label{genotyping}

Genotypes at 143,259 markers was performed using the Mouse Universal
Genotyping Array (GigaMUGA) \cite{pmid26684931} at Neogen (Lincoln, NE)
as described previously \cite{pmid29567659, pmid22345611}. Genotypes
were converted to founder strain-haplotype reconstructions using the
R/DOQTL software \cite{pmid25237114} and interpolated onto a grid with
0.02-cM spacing to yield 69,005 pseudomarkers. Individual chromosome
(Chr) haplotypes were reconstructed from RNA-seq data using a hidden
Markov model \DIFaddbegin \DIFadd{\mbox{%DIFAUXCMD
\cite{choi2020genotype} }\hskip0pt%DIFAUXCMD
}\DIFaddend (GBRS,
\url{https://github.com/churchill-lab/gbrs}). \DIFaddbegin \DIFadd{Using both methods to call
haplotypes provided redundancy for quality control. Three mice had
inconsistent calls between the two methods and were excluded from the
analysis \mbox{%DIFAUXCMD
\cite{pmid29567659}}\hskip0pt%DIFAUXCMD
.
}\DIFaddend 

\subsection{Processed \DIFaddbegin \DIFadd{DO }\DIFaddend Data}\DIFdelbegin %DIFDELCMD < \label{processed-data}
%DIFDELCMD < %%%
\DIFdelend \DIFaddbegin \label{processed-do-data}
\DIFaddend 

\DIFaddbegin \DIFadd{The DO data used in this study were generated in a previous
study\mbox{%DIFAUXCMD
\cite{pmid31343992, pmid29567659}}\hskip0pt%DIFAUXCMD
. }\DIFaddend We downloaded genotypes,
phenotypes, and pancreatic islet gene expression data from Dryad
(\url{doi:10.5061/dryad.pj105}).

\subsection{Collaborative cross recombinant inbred mice
\DIFaddbegin \DIFadd{(CC-RIX)}\DIFaddend }\DIFdelbegin %DIFDELCMD < \label{collaborative-cross-recombinant-inbred-mice}
%DIFDELCMD < %%%
\DIFdelend \DIFaddbegin \label{collaborative-cross-recombinant-inbred-mice-cc-rix}
\DIFaddend 

Mice were cared for and treated following the guidelines approved by the
Association for Assessment and Accreditation of Laboratory Animal Care
at The Jackson Laboratory. All animals were obtained from The Jackson
Laboratory. The mice were kept in a pathogen-free room at a temperature
ranging from 20 to 22°C with a 12-hour light/dark cycle. Starting at 6
weeks of age, they were fed either a custom-designed high-fat,
high-sugar (HFHS) diet (Research Diets D19070208) or a control diet
(Research Diets D19072203) \textit{ad libitum}. Body weight was measured
weekly until the mice were about 16 weeks old, after which measurements
were taken every other week. Food intake measurements were collected at
14 weeks, 23 weeks (for 6-month cohorts), 26 weeks (for 12-month
cohorts), 38 weeks, and 51 weeks by weighing the grain contents in the
cage over a three-day period. Fasted serum was collected at 14 weeks, 28
weeks (for 6-month cohorts), 26 weeks (for 12-month cohorts), 38 weeks,
and 56 weeks of age via retro-orbital or submental vein. \DIFdelbegin \DIFdel{In a subset of
the 12-month cohort, metformin (5 mg/mL) was administered
}\textit{\DIFdel{ad libitum}} %DIFAUXCMD
\DIFdel{in the drinking water.
The first two weeks of
treatment included monitoring water intake to ensure consumption.
Metformin treatment continued for the duration of the experiment.
}\DIFdelend \DIFaddbegin \DIFadd{Sex, diet, and
age were used as covariates in all analyses.
}\DIFaddend 

\subsection{Clinical chemistries}\label{clinical-chemistries}

CC-RIX animals were fasted for four hours before serum collection via
the retro-orbital or submental vein. Whole blood was left at room
temperature for 30-60 minutes before being centrifuged for 5 minutes at
12,500 RPM. The serum was then tested for glucose (Beckman Coulter;
OSR6121), cholesterol (Beckman Coulter; OSR6116), triglycerides (Beckman
Coulter; OSR60118), insulin (MSD; K152BZC-1), or c-peptide (MSD;
K1526JK-1).

\subsection{Intraperitoneal glucose tolerance
testing}\label{intraperitoneal-glucose-tolerance-testing}

After a fasting period of 4-6 hours, baseline glucose measurements were
taken from CC-RIX mice using an AlphaTrak2 glucometer and test strips
(Zoetis) by making a small nick in the tail tip. A bolus intraperitoneal
injection of 20\% glucose (1g/kg) was then administered, and additional
tail tip nicks were performed at 15, 30, 60, and 120 minutes
post-injection to measure glucose levels.

\subsection{Dual Energy X-ray Absorptiometry
(DEXA)}\label{dual-energy-x-ray-absorptiometry-dexa}

To assess bone mineral density in the CC-RIX population at either 27
weeks of age (6-month cohorts) or 55 weeks of age (12-month cohorts),
the mice were weighed and anesthetized through continuous inhalation of
isoflurane. The Faxitron UltraFocus DXA system was used to emit two
energy levels, 40 kV and 80 kV, for capturing images of bone and soft
tissue.

\subsection{Bulk tissue collection}\label{bulk-tissue-collection}

At either 28 weeks of age (for the 6-month cohort) or 56 weeks of age
(for the 12-month cohort), CC-RIX animals were humanely euthanized by
cervical dislocation. Tissues, including adipose, gastrocnemius, and the
left liver lobe, were harvested and flash-frozen in liquid nitrogen for
RNA sequencing.

\subsection{Whole Pancreas Insulin
Content}\label{whole-pancreas-insulin-content}

The animals were humanely euthanized at 16 weeks of age and the entire
pancreas was removed, ensuring no excess fat or mesentery tissue was
included. The pancreas tissue was placed in a pre-weighed 20 mL glass
scintillation vial containing acid ethanol (75\% HPLC grade ethanol
(ThermoFisher; A995-4), 1.\textbackslash5\% concentrated hydrochloric
acid (ThermoFisher; A144-212) in distilled water). The weight of the
pancreas was measured for normalization. Using curved scissors, the
pancreas was chopped for four minutes, and the samples were stored at
\(-20^{\circ}\)C until all animals were harvested. For insulin
measurements, the contents of the scintillation vials were rinsed with 4
mL PBS (Roche; 1666789) with 1\% BSA (Sigma; A7888), neutralized with 65
\si{\micro\liter} 10N NaOH (Fisher; SS255-1), and vortexed for 30
seconds. The samples were then centrifuged at \(4^{\circ}\)C for 5
minutes at 2,000 RPM. The samples were diluted 5000X in PBS with 1\%
BSA, and insulin was measured (MSD; K152BZC-1).

\subsection{RNA isolation and QC}\label{rna-isolation-and-qc}

RNA from both DO and CC-RIX adipose, gastrocnemius, and left liver lobe
tissues was isolated using the MagMAX mirVana Total RNA Isolation Kit
(ThermoFisher; A27828) and the KingFisher Flex purification system
(ThermoFisher; 5400610). The frozen tissues were pulverized with a
Bessman Tissue Pulverizer (Spectrum Chemical) and homogenized in
TRIzol\DIFdelbegin \DIFdel{™
}\DIFdelend \DIFaddbegin \DIFadd{\textsuperscript{\texttrademark} }\DIFaddend Reagent (ThermoFisher; 15596026)
using a gentleMACS dissociator (Miltenyi Biotec Inc). After adding
chloroform to the TRIzol homogenate, the RNA-containing aqueous layer
was extracted for RNA isolation, following the manufacturer's protocol,
starting with the RNA bead binding step using the RNeasy Mini kit
(Qiagen; 74104). RNA concentrations and quality were assessed using the
Nanodrop 8000 spectrophotometer (Thermo Scientific) and the RNA 6000
Pico or RNA ScreenTape assay (Agilent Technologies).

\subsection{Library construction}\label{library-construction}

Before library construction, 2 \si{\micro\liter} of diluted (1:1000)
ERCC Spike-in Control Mix 1 (ThermoFisher; 4456740) was added to 100 ng
of each RNA sample. Libraries were then constructed using the KAPA mRNA
HyperPrep Kit (Roche Sequencing Store; KK8580) following the
manufacturer's protocol. The process involves isolating polyA-containing
mRNA using oligo-dT magnetic beads, fragmenting the RNA, synthesizing
the first and second strands of cDNA, ligating Illumina-specific
adapters with unique barcode sequences for each library, and performing
PCR amplification. The quality and concentration of the libraries were
evaluated using the D5000 ScreenTape (Agilent Technologies) and the
Qubit dsDNA HS Assay (ThermoFisher; Q32851), respectively, according to
the manufacturers' instructions.

\subsection{Sequencing}\label{sequencing}

Libraries were sequenced on an Illumina NovaSeq 6000 using the S4
Reagent Kit (Illumina; 20028312). All tissues underwent 100 bp
paired-end sequencing, aiming for a target read depth of 30 million read
pairs.

\subsection{Trait selection in DO}\label{trait-selection-in-do}

We filtered the measured traits in this study to a set of relatively
non-redundant measures that were well-represented in the population
(having at least 80\% of individuals measured). A complete description
of trait filtering can be found at Figshare DOI:
10.6084/m9.figshare.27066979 in the file Documents \textgreater{} 1.DO
\textgreater{} 1b.Trait\_Selection.Rmd.

We took two approaches for traits with multiple redundant measurements,
for example longitudinal body weights. In the case of longitudinal
measurements, we used the final measurement, as this was the closest
physiological measurement to the measurement of gene expression, which
was done at the end of the experiment. The labels for these traits have
the word ``Final'' appended to their name. For traits with multiple
highly related measurements, such as cholesterol, we used the first
principal component of the group of measurements. For example, we used
the first principal component of all LDL measurements as the measurement
of LDL. For each set of traits, we ensured the first principal component
had the correct sign by correlating it with the average of the traits.
For correlation coefficients (R) less than 0, we multiplied the
principal component by -1. The labels for these traits have the term
``PC1'' appended to their name.

\subsection{Processing of RNA sequencing
data}\label{processing-of-rna-sequencing-data}

We used the Expectation-Maximization algorithm for Allele Specific
Expression (EMASE) \cite{pmid29444201, pmid25236449} to quantify
multi-parent allele-specific and total expression from RNA-seq data for
each tissue. EMASE was performed by the Genotype by RNA-seq (GBRS)
software package (\url{https://gbrs.readthedocs.io/en/latest/}). In the
process, R1 and R2 FASTQ files were combined and aligned to a hybridized
(8-way) transcriptome generated for the 8 DO founder strains as
single-ended reads. GBRS was also used to reconstruct the mouse genotype
probabilities along \DIFdelbegin \DIFdel{\textasciitilde69K }\DIFdelend \DIFaddbegin \DIFadd{\(\sim69\)K }\DIFaddend markers, which was used for confirming
genotypes in the quality control (QC) process. For the QC process, we
used a Euclidean distances method (developed by Greg Keele - Churchill
Lab) to compare the GBRS genotype probabilities between the tissues and
the genotype probabilities array for all mice. The counts matrix for
each tissue was processed to filter out transcripts with less than one
read for at least half of the samples. RNA-seq batch effects were
removed by regressing out batch as a random effect and considering sex
and generation as fixed effects using lme4 R package. RNA-Seq counts
were normalized relative to total read counts using the variance
stabilizing transform (VST) as implemented in DESeq2 and using rank
normal score.

\subsection{eQTL analysis}\label{eqtl-analysis}

We used R/qtl2 \cite{pmid30591514} to perform eQTL analysis. We used the
rank normal score data and used sex and DO generation as additive
covariates. We also used kinship as a random effect. We used
permutations to find a LOD threshold of 8 for significant QTLs which
corresponded to a \DIFaddbegin \DIFadd{nominal }\DIFaddend \(p\) value of 0.05.

To assess whether eQTL were shared across tissues, we considered
significant eQTLs within 4Mb of each other to be overlapping. We
considered local and distal eQTLs separately. \DIFaddbegin \DIFadd{Local eQTL were defined as
an eQTL within 4Mb of the transcription start site of the encoding gene.
}\DIFaddend 

\subsection{Local and distal heritability of
transcripts}\label{local-and-distal-heritability-of-transcripts}

To estimate local and distal heritability of each transcript, we scaled
each normalized transcript to have a variance of 1. We then modeled this
transcript with the local genotype using the fit1() function in R/qtl\DIFdelbegin \DIFdel{with a kinship correction}\DIFdelend .
We used the resulting model to predict the transcript values. The
variance of the predicted transcript is its local heritability. We then
estimated the heritability of the residual of the model fit. The
variance of the residual multiplied by its heritability is the distal
heritability of the transcript.

We compared local and distal estimates of heritability to measures of
trait relevance for each transcript. \DIFdelbegin \DIFdel{Trait relevance was the Pearson
correlation
(R) between the transcript and the trait }\DIFdelend \DIFaddbegin \DIFadd{To calculate trait relevance of a
given transcript, we adjusted normalized transcript values for sex, DO
wave, and DO generation. We similarly adjusted traits by sex, DO wave,
and DO generation. We then calculated all Spearman correlation
coefficients (\(\rho\)) between adjusted traits and adjusted
transcripts. The trait relevance of a given tanscript was the maximum
absolute correlation coefficient across all traits}\DIFaddend .

\subsection{High-dimensional mediation
analysis}\label{high-dimensional-mediation-analysis}

In this section we derive the objective function for high-dimensional
mediation analysis (HDMA) and present an iterative algorithm to optimize
this objective function. Our starting point is the univariate case,
where we describe perfect mediation as a constraint on the covariance
matrix among variables. We then leverage this constraint to define
projections of multivariate data that are maximally consistent with
perfect mediation (HDMA). Next, we demonstrate how to \textit{kernelize}
HDMA to limit dimensionality of the model and enable non-linear HDMA
models.

\subsubsection{Perfect mediation as a constraint on covariance
matrices}\label{perfect-mediation-as-a-constraint-on-covariance-matrices}

Suppose we have three random variables \(x\), \(m\), and \(y\). Assume
they each have unit variance and that they satisfy the following
structural equation model (SEM) such that \(m\) perfectly mediates the
effect of \(x\) on \(y\):

\begin{align}
m &= \alpha x + \epsilon_m \label{eqn:perfect_mediation1}\\ 
y &= \beta m + \epsilon_y  \label{eqn:perfect_mediation2}
\end{align}

From these structural equations, we have the
\textit{model-implied covariance matrix}, \(\Sigma\), given by

\begin{align}
\label{eqn:model_implied_covariance}
\Sigma = 
  \begin{bmatrix}
1 & \alpha & \alpha \beta \\
\alpha & 1 & \beta \\
\alpha \beta & \beta & 1
\end{bmatrix}
\end{align} Note that the assumption of perfect mediation forces the
covariance between \(x\) and \(y\) to be \(\alpha \beta\). In any finite
data set, however, the observed covariance matrix, \(S = [S_{ij}]\),
will not typically satisfy this constraint.

The general negative log-likelihood fitting function for an SEM is given
by

\begin{align}
\label{eq:loglikelihood}
L &= {\rm tr}\left(S \Sigma^{-1}\right) + \log\left|\Sigma\right|,
\end{align} where \(|\cdot|\) denotes the determinant of a matrix and
\({\rm tr}(\cdot)\) denotes the trace \cite{bollen2014structural}. For
the perfect-mediation model, these values are

\begin{align}
\left|\Sigma\right| &= (1-\alpha^2)(1-\beta^2) \\
%S^{-1} &= 
  %\begin{bmatrix}
%\frac{1}{1-\alpha^2} & \frac{-\alpha}{1-\alpha^2} & 0 \\
%\frac{-\alpha}{1-\alpha^2} & \frac{1-\alpha^2\beta^2}{(1-\alpha^2)(1-%\beta^2} & \frac{-\beta}{1-\beta^2} \\
%0 & \frac{-\beta}{1-\beta^2} & \frac{1}{1-\beta^2}
%\end{bmatrix}\\ 
\Sigma^{-1} &= 
  \begin{bmatrix}
1/(1-\alpha^2) & -\alpha/(1-\alpha^2) & 0 \\
-\alpha/(1-\alpha^2) & (1-\alpha^2\beta^2)/\left((1-\alpha^2)(1-\beta^2)\right) & -\beta/(1-\beta^2) \\
0 & -\beta/(1-\beta^2) & 1/(1-\beta^2)
\end{bmatrix}
\end{align} Plugging these into the likelihood function, we get
\begin{align}
\label{eqn:mediation_loglikelihood}
L = &\log\left((1-\alpha^2)(1-\beta^2)\right) - \frac{2\alpha^2\beta^2}{(1-\alpha^2)(1-\beta^2)} + 1 - \frac{2\alpha}{1-\alpha^2} S_{12} - \frac{2\beta}{1-\beta^2} S_{23}
\end{align} To simplify notation, we define \begin{equation}
F(\alpha,\beta) = \log\left((1-\alpha^2)(1-\beta^2)\right) - \frac{2\alpha^2\beta^2}{(1-\alpha^2)(1-\beta^2)} + 1,
\end{equation} so the likelihood function is now \begin{align}
\label{eqn:mediation_loglikelihood_simple}
L = F(\alpha, \beta) - \frac{2\alpha}{1-\alpha^2} S_{12} - \frac{2\beta}{1-\beta^2} S_{23}
\end{align} Note that this likelihood is maximized by fitting regression
coefficients \(\alpha\) and \(\beta\) between \(x\) and \(m\) and \(m\)
and \(y\), respectively, but the negative log-likelihood formulation is
useful for the multivariate extension below.

\subsubsection{Projecting multivariate data to identify latent
mediators}\label{projecting-multivariate-data-to-identify-latent-mediators}

Suppose now that we have three data matrices, \(X\), \(M\), and \(Y\)
(individuals by variables) that are mean centered by column. The central
assumption of HDMA is that these multivariate data encode
\textit{latent variables} that are causally linked according to the
perfect-mediation model, in a sense made precise as follows.

We use the log-likelihood function (Eqn.
\ref{eqn:mediation_loglikelihood}) of the perfect mediation model as an
objective function to identify latent variables, \(l_X\), \(l_M\), and
\(l_Y\), that are are correlated as closely as possible to the
constraints of the perfect mediation model, Eqn.
(\ref{eqn:model_implied_covariance}). We estimate these latent variables
as linear combinations of the measured variables

\begin{align}
l_X &= Xa \\
l_M &= Mb \\
l_Y &= Yc
\end{align}

The coefficient vectors \(a\), \(b\), and \(c\), are called
\textit{loadings}, analogous to the terminology in PCA and CCA. Because
the data matrices are mean centered, we have

\begin{equation}
{\rm mean}(l_X) = {\rm mean}(l_M) = {\rm mean}(l_Y) = 0,
\end{equation}

and we assume the loadings are scaled so that each latent variable has
unit variance

\begin{equation}
{\rm var}(l_X) = {\rm var}(l_M) = {\rm var}(l_Y) = 1.
\end{equation}

Plugging these formulae into the objective function (Eqn.
\ref{eqn:mediation_loglikelihood_simple}), we have

\begin{align}
S_{12} &= {\rm corr}\left(l_X, l_M\right)\\
S_{23} &= {\rm corr}\left(l_M, l_Y\right)\\
\label{eqn:gcca_mediation}
L(\alpha, \beta, a,b,c) &= F(\alpha, \beta) - \frac{2\alpha}{1-\alpha^2} {\rm corr}\left(l_X, l_M\right) - \frac{2\beta}{1-\beta^2} {\rm corr}\left(l_M, l_Y\right) \\ 
&= F(\alpha, \beta) - \frac{2\alpha}{1-\alpha^2} {\rm corr}\left(Xa, Mb\right) - \frac{2\beta}{1-\beta^2} {\rm corr}\left(Mb, Yc\right)
\end{align}

This yields an objective function of two sets of parameters: the
\textit{structural parameters} \(\alpha\) and \(\beta\) that define the
causal model among latent variables, and the loading vectors \(a\),
\(b\), and \(c\), that define the latent variables in terms of the
measured variables. The goal of HDMA is to optimize \(L\) as a function
of all parameters simultaneously. The form of the objective function,
Eqn. \ref{eqn:gcca_mediation}, is effectively a weighted sum of
correlation coefficients, connecting it to so-called
\textit{sum-of-correlation}, or SUMCOR, optimization problems
\cite{tenenhaus2011regularized}, which we discuss further below.

\subsubsection{An algorithm for HDMA}\label{an-algorithm-for-hdma}

The global optimization of \ref{eqn:gcca_mediation} is challenging
because it is not a convex problem. However, the decomposition of the
variables into structural and loading variables suggests an iterative
algorithm, similar to the expectation-maximization algorithm, that
converges at least to a stationary point. The overall idea is to use a
block-coordinate-ascent strategy that iterates between optimizing \(a\),
\(b\), and \(c\), then optimizing \(\alpha\) and \(\beta\).

For fixed \(a\), \(b\), and \(c\), the optimal \(\alpha\) and \(\beta\)
are simply given by regression coefficients between \(l_X\) and \(l_M\)
and \(l_M\) and \(l_Y\), respectively. Given these regression
coefficients, \(\alpha\) and \(\beta\), we then optimize \(a\), \(b\),
and \(c\). For fixed \(\alpha\) and \(\beta\), the term
\(F(\alpha, \beta)\) is irrelevant, so minimizing the negative
log-likelihood function reduces to maximizing the reduced function

\begin{align}
L_{red}(a,b,c) = \frac{2\alpha}{1-\alpha^2} {\rm corr}\left(Xa, Mb\right) + \frac{2\beta}{1-\beta^2} {\rm corr}\left(Mb, Yc\right),
\end{align}

which is a weighted sum of correlation coefficients. This is exactly a
(weighted) SUMCOR optimization problem \cite{tenenhaus2011regularized}.
These optimization problems are still not convex, but Tenenhaus
\textit{et al.} have recently proved convergence for iterative
algorithms that optimize weighted SUMCOR problems
\cite{tenenhaus2011regularized,
tenenhaus2017regularized, tenenhaus2015kernel}. These algorithms only
guarantee convergence to a stationary point not necessarily a maximum,
as is common in other non-convex problems, but this can be overcome with
multiple random restarts, if needed. Thus, we have a sub-routine
\(\texttt{wSUMCOR}(X, M, Y, w_1, w_2)\) that solves the weighted SUMCOR
problem

\begin{align}
L_{wSUMCOR}(a,b,c, w_1, w_2) = w_1 {\rm corr}\left(Xa, Mb\right) + w_2 {\rm corr}\left(Mb, Yc\right).
\end{align}

Iterating between optimizing the structural parameters and loading
parameters, we reduce the negative log-likelihood at each step and
converge to a fixed point.

We summarize our optimization procedure in Algorithm \ref{alg:hdma}.

\begin{algorithm}
\caption{High-dimensional mediation analysis}\label{alg:hdma}
\begin{algorithmic}
\Require $X$, $M$, $Y$ \Comment{Data matrices}
\Ensure $\alpha$, $\beta$, $a$, $b$, $c$, $l_X$, $l_M$, $l_Y$ \Comment{Structural parameters, loadings, scores}
%\begin{itemize}
%    \item $\alpha$, $\beta$ \Comment{Structural parameters}
%    \item $a$, $b$, $c$ \Comment{Loadings}
%    \item $l_X = Xa$, $l_M = Mb$, $l_Y = Yc$ \Comment{Scores}
%\end{itemize}
\State $\alpha \gets 0.5$, $\beta \gets 0.5$ \Comment{Initialize structural parameters}
\While{$converge \neq TRUE$}
\State $d \gets \frac{2\alpha}{1-\alpha^2} + \frac{2\alpha}{1-\alpha^2}$ \Comment{Normalization constant for weights}
\State $w_1 \gets \frac{1}{d}\frac{2\alpha}{1-\alpha^2}$, $w_2 \gets \frac{1}{d}\frac{2\beta}{1-\beta^2}$ \Comment{Set weights (sum to one)}
\State $(a,b,c) \gets \texttt{wSUMCOR}(X, M, Y, w_1, w_2)$  \Comment{Compute loadings}
\State $l_X \gets Xa$, $l_M \gets Mb$, $l_Y \gets Yc$ \Comment{Compute scores}
\State $\alpha \gets {\rm corr}(l_X, l_M)$, $\beta \gets {\rm corr}(l_M, l_Y)$ \Comment{Update structural parameters}
\EndWhile
\end{algorithmic}
\end{algorithm}

\subsubsection{Kernel HDMA}\label{kernel-hdma}

For large data matrices \(X\), \(M\), and \(Y\), especially with high
correlation among variables, as is common for high-throughput biological
assays (\textit{e.g.}, \textasciitilde1M alleles for genotypes,
\textasciitilde20k transcripts), we can further reduce the
dimensionality of the HDMA model by requiring that loading vectors lie
in the span of the the measured individuals, namely

\begin{align}
a &= X^T \tilde{a} \\
b &= M^T \tilde{b} \\
c &= Y^T \tilde{c}.
\end{align}

This replaces the full feature data, say \(X\), with the covariances
among individuals (aka, Gram matrices), \(C_X = XX^T\), and reduces the
dimensionality from the number of measured variables down to the number
of individuals

\begin{align}
l_x &= XX^T \tilde{a} = C_X \tilde{a} \\
l_M &= MM^T \tilde{b} = C_M \tilde{b} \\
l_M &= YY^T \tilde{c} = C_Y \tilde{c}.
\end{align}

This reduction is called \textit{kernelization}
\cite{tenenhaus2015kernel} and is widely applied to other linear models,
including CCA, linear regression, and classification.

It is interesting to note that kernelization is often used to convert a
linear model to a non-linear model by replacing the covariance matrices,
\textit{e.g.} \(C_X\), with more complex \textit{kernel matrices}
\(K_X\) that encode similarity measures among individuals that are
non-linear functions of the measured variables. non-linear model by
replacing the covariance matrices, \textit{e.g.} \(C_X\), with more
complex \textit{kernel matrices} \(K_X\) that encode similarity measures
among individuals that are non-linear functions of the measured
variables. Promoting a linear model to a non-linear model in this way is
called the \textit{kernel trick} and is widely used in the machine
learning field. The above considerations show that HDMA is kernelizable
in the same way as other linear models, although the exploration of
non-linear models is outside the scope of this study.

\subsubsection{Implementation details}\label{implementation-details}

We have implemented HDMA (Algorithm \ref{alg:hdma}) in the R programming
language. Tenenhaus \textit{et al.} have implemented their optimizers in
the Regularized Generalized Canonical Correlation Analysis (RGCCA) R
package \cite{girka2023multiblock}, which we use as the subroutine
\texttt{wSUMCOR}. As Tenenhaus \textit{et al.} discuss optimizing the
empirical correlation coefficient \textit{per se} is numerically
unstable due to the inversion of the covariance matrices of the measured
variables (\textit{e.g.}, the transcript-transcript covariance matrix).
To overcome this, the RGCCA package uses a regularized form of the
covariance matrix developed by Schaeffer and Strimmer
\cite{schafer2005shrinkage}, which can be estimated rapidly using an
analytic formula.

As a convergence criterion, we stop the iterations when both \(\alpha\)
and \(\beta\) change by less than \(10^{-6}\) from their previous value
in one iteration.

All code required to run HDMA is available at Figshare:
\url{https://figshare.com/} DOI: 10.6084/m9.figshare.27066979

\subsection{Enrichment of biological
terms}\label{enrichment-of-biological-terms}

We performed gene set enrichment analysis (GSEA) \cite{pmid16199517}
using the transcript loadings in each tissue as gene weights. GSEA
determines enrichment of pathways based on where the contained genes
appear in a ranked list of genes. If the genes in the pathway are more
concentrated near the top (or the bottom) of the list than expected by
chance, the pathway can be interpreted as being enriched with positively
(negatively) loaded transcripts. We used the R package fgsea
\cite{fgsea} to calculate normalized enrichment scores for all GO terms
and all KEGG pathways.

We downloaded all KEGG \cite{pmid36300620} pathways for
\textit{Mus musculus} using the R package clusterProfiler
\cite{pmid36300620}. We then used fgsea to calculate enrichment scores
in each tissue using the transcript loadings in each tissue as our
ranked list of genes. We reported the normalized enrichment score (NES)
for the 10 pathways with the largest positive NES and the 10 pathways
with the largest negative NES.

We used the R package pathview \cite{pmid23740750} to visualize the
loadings from each tissue in interesting pathways. We scaled the
loadings in each tissue by the maximum absolute value of loadings across
all tissues to compare them across tissues.

We downloaded GO term annotations from Mouse Genome Informatics at the
Jackson Laboratory \cite{pmid33231642}
\url{https://www.informatics.jax.org/downloads/reports/index.html} We
removed gene-annotation pairs labeled with NOT, indicating that these
genes were known not to be involved in these GO terms. We also limited
our search to GO terms with between 80 and 3000 genes. We used the R
package annotate \cite{R_annotate} to identify the ontology of each term
and the R package pRoloc \cite{pmid24413670} to convert between GO terms
and names. As with the KEGG pathways, we used fgsea to calculate a
normalized enrichment score for each GO term and collected loadings for
the transcripts in each term to compare across tissues.

\subsection{TWAS in DO mice}\label{twas-in-do-mice}

We performed a transcriptome-wide analysis (TWAS) \cite{pmid26258848, 
pmid26854917} in the DO mice to compare to the results of
high-dimensional mediation. To perform TWAS, we fit a linear model to
explain variation in each transcript across the population using the
genotype at the nearest marker to the gene transcription start site
(TSS). We used kinship as a random effect and sex, diet, and DO
generation as fixed effects. The predicted transcript from each of these
models was the imputed transcript based only on the local genotype.

We correlated each imputed transcript with each of the metabolic
phenotypes after adjusting phenotypes for sex, diet, and DO generation.
To calculate significance of these correlations, we performed
permutation testing by shuffling labels of individual mice and
recalculating correlation values. Significant correlations were those
more extreme than any of the permuted values, corresponding to an
empirical \(p\) value of 0. These are transcripts whose locally encoded
expression level was significantly correlated with one of the metabolic
traits. This suggests an association between the genetically encoded
transcript level and the trait but does not identify a direction of
causation.

\subsection{Literature support for
genes}\label{literature-support-for-genes}

To determine whether each gene among those with large loadings or large
heritability had a supported connection to obesity or diabetes in the
literature, we used the R package easyPubMed \cite{easyPubMed}. We
searched for the terms (``diabetes'' OR ``obesity'') along with the
tissue name (adipose, islet, liver, or muscle), and the gene name. We
restricted the gene name to appear in the title or abstract as some
short names appeared coincidentally in contact information. We checked
each gene with apparent literature support by hand to verify that
support, and we removed spurious associations. For example, FAU is used
as an acronym for fatty acid uptake and CAD is used as an acronym for
coronary artery disease. Both terms co-occur with the terms diabetes and
obesity in a manner independent of the genes \textit{Fau} and
\textit{Cad}. Other genes that co-occurred with diabetes and obesity,
but not as a functional connection were similarly removed. For example,
the gene \textit{Rpl27} is used as a reference gene for quantification
of the expression of other genes, and co-occurrence with diabetes and
obesity is a coincidence. We counted the abstracts associated with
diabetes or obesity and each gene name and determined that a gene had
literature support when it had at least two abstracts linking it to the
terms diabetes or obesity in the respective tissue.

\subsection{Tissue-specific clusters}\label{tissue-specific-clusters}

To compare the top loading genes across tissues, we selected genes with
a loading at least 2.5 standard deviations from the mean across all
tissues. We made a matrix consisting of the union of these sets
populated with the tissue-specific loading for each gene. We used the
pam() function in the R package cluster \cite{Rcluster} to cluster the
loading profiles around \(k\) medoids. We tested \(k = 2\) through 20
and used silhouette andlysis to compare the separation of the clusters.
The best separation was achieved with \(k = 12\) clusters. For each
cluster we used the R package gprofiler2 \cite{Rgprofiler2} to identify
enriched GO terms and KEGG pathways for the genes in each cluster.

\subsection{CC-RIX genotypes}\label{cc-rix-genotypes}

We used the most recent common ancestor (MRCA) genotypes for the
Collaborative Cross (CC) mice available on the University of North
Carolina Computational Systems Biology website:
\url{http://www.csbio.unc.edu/CCstatus/CCGenomes/}

To generate CC-RIX genotypes, we averaged the haplotype probabilities
for the two parental strains at each locus.

\subsection{Imputation of gene expression in
CC-RIX}\label{imputation-of-gene-expression-in-cc-rix}

To impute gene expression in the CC-RIX, we performed the following
steps for each transcript in each tissue (adipose, liver, and skeletal
muscle):

\begin{enumerate}
\def\labelenumi{\arabic{enumi}.}
\tightlist
\item
  Calculate diploid CC-RIX genotype for all CC-RIX individuals at the
  marker nearest the transcription start site of the transcript.
\item
  Multiply the genotype probabilities by the eQTL coefficients
  identified in the DO population.
\end{enumerate}

To check the accuracy of the imputation, we correlated each imputed
transcript with the measured transcript. The average Pearson correlation
(r) was close to 0.5 for all three tissues (Supp. Fig. S7A), and as
expected, the correlation between the imputed transcript and the
measured transcript was highly positively dependent on the local eQTL
LOD score of the transcript (Supp. Fig. S7B).

\subsection{Prediction of CC-RIX
traits}\label{prediction-of-cc-rix-traits}

We used both measured expression and imputed expression combined with
the results from HDM in the to predict phenotype in the CC-RIX. The
traits measured in the DO and the CC-RIX were not identical, so we
limited our prediction to body weight, which was measured in both
populations, and was the largest contributor to the phenotype score in
the DO.

For each CC-RIX individual, we multiplied the transcript abundances
across the transcriptome by the loadings derived from the HDM in the DO
population (Fig. 7A). This resulted in a vector with \(n\) elements,
where \(n\) is the number of transcripts in the trancriptome. Each
element was a weighted value that combined the relative abundance of the
transcript with how that abundance affected the phenotype. We averaged
the values in this vector to calculate an overall predicted phenotype
score for the individual CC-RIX animal.

After calculating this predicted phenotype value across all CC-RIX
animals, we correlated the predicted values from each tissue with
measured body weight (Fig. 7B).

\subsection{Cell type specificity}\label{cell-type-specificity}

We investigated whether the loadings derived from HDM reflected tissue
composition changes in the DO mice prone to obesity on the high-fat
diet. To do this, we acquired lists of cell-type specific transcripts
from the literature. In adipose tissue, we looked at cell-type specific
transcripts for macrophages, leukocytes, adipocyte progenitors, and
adipocytes as defined in {[}29087381{]}. In pancreatic islets, we looked
at cell-type specific transcripts for alpha cells, beta cells, delta
cells, ductal cells, mast cells, macrophages, acinar cells, stellate
cells, gamma and epsilon cells, and endothelial cells as defined by
{[}36778506{]}. Both studies defined cell-type specific transcripts
based on human cell types. We collected the loadings for each set of
cell-type specific transcripts in the respective tissue and asked
whether the mean loading for the cell type differed significantly from 0
(Fig. 8). A significant positive loading for the cell type would suggest
a genetic predisposition to have a higher proportion of that cell type
in the tissue. To determine whether each mean loading differed
significantly from 0, we performed permutation tests. We randomly
sampled \(n\) genes outside of the cell-type specific, where \(n\) was
the number of genes in the set. We compared the distribution of loading
means over 10,000 random draws to that seen in the observed data. We
used a significance threshold of 0.01.

\subsection{Comparison of transcriptomic signatures to human
transcriptomic
signatures}\label{comparison-of-transcriptomic-signatures-to-human-transcriptomic-signatures}

To compare the transcriptomic signatures identified in the DO mice to
those seen in human patients, we downloaded human gene expression data
from the Gene Expression Omnibus (GEO)
\cite{pmid37933855, pmid11752295}. We focused on adipose tissue because
this had the strongest relationship to obesity and insulin resistance in
the DO. We downloaded the following human gene expression data sets:

\begin{itemize}
\item
  Accession number GSE152517 - Performed bulk RNA sequencing on visceral
  adipose tissue resected from seven diabetic and seven non-diabetic
  obese individuals.
\item
  Accession number GSE44000 - Used Agilent-014850 4X44K human whole
  genome platform arrays (GPL6480) to measure gene expression in
  purified adipocytes derived from the subcutaneous adipose tissue of
  seven obese (BMI\textgreater30) and seven lean (BMI\textless25)
  post-menopausal women.
\item
  Accession number GSE205668 - Subcutaneous adipose tissue was resected
  during elective surgery from 35 normal weight, and 26 obese children.
  Gene expression was measured by RNA sequencing with an Illumina HiSeq
  2500.
\item
  Accession number GSE29231 - Visceral adipose biopsies were taken from
  three female patients with type 2 diabetes, and three non-diabetic
  female patients. Expression was measured with Illumina HumanHT-12 v3
  Expression BeadChip arrays.
\end{itemize}

We downloaded each data set from GEO using the R package GEOquery
\cite{geoquery}. In each case, we verified that gene expression was log
transformed and performed the transformation ourselves if it had not
already been done. When covariates such as age and sex were available in
the metadata files, we regressed out these variables. We mean centered
and standardized gene expression across transcripts.

We matched the human gene expression to the mouse gene expression by
pairing orthologs as defined in The Jackson Laboratory's mouse genome
informatics data base (MGI) \cite{pmid38531069}. We multiplied each
transcript in the human data by the adipose tissue loading of its
ortholog in the DO mice. This resulted in a vector of weighted
transcript values for each patient based on their own transcriptional
profile and the obesity-related transcriptional signature from the DO
analysis. The mean of this vector for an individual was the prediction
of their obesity status. Higher values indicate a prediction of higher
obesity or risk of metabolic disease based on adipose gene expression.
We then compared the values across groups, either obese and non-obese,
or diabetic and non-diabetic depending on the groups in each study.

\subsection{Connectivity Map Queries}\label{connectivity-map-queries}

We queried the transcript loading signatures from adipose tissue and
pancreatic islets with the CMAP database. These tissues are the most
related to metabolic disease and diabetes respectively.

The gene expression profiles in the Connectivity Map database are
derived from human cell lines and human primary cultures and are indexed
by Entrez gene IDs. To query the CMAP database, we identified the Entrez
gene IDs for the human orthologs of the mouse genes expressed in each
tissue. Each CMAP query takes the 150 most up-regulated and the 150 most
down-regulated genes in a signature, however, not all human genes are
included in their database. To ensure we had as many genes as possible
in the query, we selected the top and bottom 200 genes with the most
extreme positive and negative loadings respectively. We pasted these
into the CLUE query application available at
\url{https://clue.io/query}.

We filtered the results in two ways: First, we looked at the most
significantly anti-correlated (\(-log_{10}(\mathrm{FDR} q) > 15\)) hits
across all cell types. Second, we looked at the most anti-correlated
within the most related cell type to the query and considered hits
regardless of \(-log_{10}(\mathrm{FDR} p)\). For adipose tissue we
looked in normal adipocytes, abbreviated ASC in the CMAP database, and
for pancreatic islets we looked in pancreatic cancer cells, abbreviated
YAPC in the CMAP database.

\bibliographystyle{unsrt}
\bibliography{islet.bib}

\end{document}
