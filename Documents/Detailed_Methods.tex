% Options for packages loaded elsewhere
\PassOptionsToPackage{unicode}{hyperref}
\PassOptionsToPackage{hyphens}{url}
%
\documentclass[
]{article}
\usepackage{amsmath,amssymb}
\usepackage{iftex}
\ifPDFTeX
  \usepackage[T1]{fontenc}
  \usepackage[utf8]{inputenc}
  \usepackage{textcomp} % provide euro and other symbols
\else % if luatex or xetex
  \usepackage{unicode-math} % this also loads fontspec
  \defaultfontfeatures{Scale=MatchLowercase}
  \defaultfontfeatures[\rmfamily]{Ligatures=TeX,Scale=1}
\fi
\usepackage{lmodern}
\ifPDFTeX\else
  % xetex/luatex font selection
\fi
% Use upquote if available, for straight quotes in verbatim environments
\IfFileExists{upquote.sty}{\usepackage{upquote}}{}
\IfFileExists{microtype.sty}{% use microtype if available
  \usepackage[]{microtype}
  \UseMicrotypeSet[protrusion]{basicmath} % disable protrusion for tt fonts
}{}
\makeatletter
\@ifundefined{KOMAClassName}{% if non-KOMA class
  \IfFileExists{parskip.sty}{%
    \usepackage{parskip}
  }{% else
    \setlength{\parindent}{0pt}
    \setlength{\parskip}{6pt plus 2pt minus 1pt}}
}{% if KOMA class
  \KOMAoptions{parskip=half}}
\makeatother
\usepackage{xcolor}
\usepackage[margin=1in]{geometry}
\usepackage{graphicx}
\makeatletter
\def\maxwidth{\ifdim\Gin@nat@width>\linewidth\linewidth\else\Gin@nat@width\fi}
\def\maxheight{\ifdim\Gin@nat@height>\textheight\textheight\else\Gin@nat@height\fi}
\makeatother
% Scale images if necessary, so that they will not overflow the page
% margins by default, and it is still possible to overwrite the defaults
% using explicit options in \includegraphics[width, height, ...]{}
\setkeys{Gin}{width=\maxwidth,height=\maxheight,keepaspectratio}
% Set default figure placement to htbp
\makeatletter
\def\fps@figure{htbp}
\makeatother
\setlength{\emergencystretch}{3em} % prevent overfull lines
\providecommand{\tightlist}{%
  \setlength{\itemsep}{0pt}\setlength{\parskip}{0pt}}
\setcounter{secnumdepth}{-\maxdimen} % remove section numbering

\newcommand{\beginsupplement}{%
        \setcounter{table}{0}
        \renewcommand{\thetable}{S\arabic{table}}%
        \setcounter{figure}{0}
        \renewcommand{\thefigure}{S\arabic{figure}}%
     }

\usepackage{lineno}
\linenumbers

\usepackage{natbib}

\usepackage{setspace} \doublespacing
\ifLuaTeX
  \usepackage{selnolig}  % disable illegal ligatures
\fi
\IfFileExists{bookmark.sty}{\usepackage{bookmark}}{\usepackage{hyperref}}
\IfFileExists{xurl.sty}{\usepackage{xurl}}{} % add URL line breaks if available
\urlstyle{same}
\hypersetup{
  pdftitle={Detailed Methods for High Dimensional Mediation for Massively Polygenic Traits},
  pdfauthor={J. Matthew Mahoney and Anna L Tyler},
  hidelinks,
  pdfcreator={LaTeX via pandoc}}

\title{Detailed Methods for High Dimensional Mediation for Massively
Polygenic Traits}
\author{J. Matthew Mahoney and Anna L Tyler}
\date{June 03, 2024}

\begin{document}
\maketitle

\subsection{Diversity Outbred Mice}\label{diversity-outbred-mice}

We placed a population of 500 male and female mice from generations 18,
19, and 21 on a high-fat, high-sugar diet to induce diet-associated
obesity and metabolic disease as described previously
\cite{pmid29567659}. Over the experimental period of 18 weeks multiple
metabolic traits were measured longitudinally, including body weight,
plasma levels of insulin and glucose, and plasma lipids. At the end of
the experiment, we used RNASeq to measure gene expression in 384 mice in
four tissues relevant to metabolic disease: adipose tissue, pancreatic
islets, liver, and skeletal muscle. The mice were also genotyped using
the Mouse Universal Genotyping Array (GigaMUGA).

\subsection{Trait measurements}\label{trait-measurements}

also described in \cite{pmid29567659} briefly describe: oral glucose
tolerance tests

\subsection{Genotyping}\label{genotyping}

also described in \cite{pmid29567659} briefly describe here.

\subsection{RNA Sequencing}\label{rna-sequencing}

also described in \cite{pmid29567659} describe islet picking, and how
other organs were collected. mention details about GigaMUGA

\subsection{Trait selection}\label{trait-selection}

We filtered the measured traits in this study to a set of relatively
non-redundant measures that were well-represented in the population
(having at least XXX individuals), and spanned multiple aspects of
metabolic disease. A complete description of trait filtering can be
found in File XXX (1b.Trait\_Selection.Rmd).

We took two approaches for traits with multiple redundant measurements,
for example logintudinal body weights. In the case of longitudinal
measurements, we used the final measurement, as this was the closest
physiological measurement to the measurement of gene expression, which
was done at the end of the experiment. The labels for these traits are
have the word ``Final'' appended to their name. For traits with multiple
highly related measurements, such as cholesterol, we used the first
principal component of the group of measurements. For example, we used
the first principal component of all LDL measurements as the measurement
of LDL. For each set of traits, we ensured the first principal component
had the correct sign by correlating it with the average of the traits.
For correlation coefficients (R) less than 0, we multiplied the
principal component by -1. The labels for these traits have the term
``PC1'' appended to their name.

\subsection{Processing of RNA sequencing
data}\label{processing-of-rna-sequencing-data}

We used the Expectation-Maximization algorithm for Allele Specific
Expression (EMASE) {[}cite{]} to quantify multi-parent allele-specific
and total expression from RNA-seq data for each tissue. EMASE was
performed by the Genotype by RNA-seq (GBRS) software package
(\url{https://gbrs.readthedocs.io/en/latest/}). In the process, R1 and
R2 FASTQ files were combined and aligned to a hybridized (8-way)
transcriptome generated for the 8 DO founder strains as single-ended
reads. GBRS was also used to reconstruct the mouse genotype
probabilities along \textasciitilde69K markers, which was used for
confirming genotypes in the quality control (QC) process. For the QC
process, we used a Euclidean distances method (developed by Greg Keele -
Churchill Lab) to compare the GBRS genotype probabilities between the
tissues and the genotype probabilities array for all mice. The counts
matrix for each tissue was processed to filter out transcripts with less
than one read for at least half of the samples. RNA-seq batch effects
were removed by regressing out batch as a random effect and considering
sex and generation as fixed effects using lme4 R package. RNA-Seq counts
were normalized relative to total read counts using the variance
stabilizing transform (VST) as implemented in DESeq2 and using rank
normal score.

We used R/qtl2 {[}cite{]} to perform eQTL analysis. We used the rank
normal score data and used sex and DO generation as additive covariates.
We also used kinship as a random effect. We used permutations to find a
LOD threshold of 8 for significant QTLs which corresponded to a \(p\)
value of 0.05.

To assess whether eQTL were shared across tissues, we compared eQTLs for
each transcript across tissues. Significant eQTLs within 4Mb of each
other were considered overlapping. We considered local and distal eQTLs
separately.

To estimate local and distal heritability of each transcript, we scaled
each normalized transcript to have a variance of 1. We then modeled this
transcript with the local genotype using the fit1() function in R/qtl
with a kinship correction. We used the resulting model to predict the
transcript values. The variance of the predicted transcript its local
heribatility. We then estimated the heritability of the residual of the
model fit. The variance of the residual multiplied by its heritability
is the distal heritability of the transcript.

We compared local and distal estimates of heritability to measures of
trait relevance for each transcript. Trait relevance, was the Pearson
correlation (R) between the transcript and the trait.

\subsection{Kernelization}\label{kernelization}

Before running high-dimensional mediation analysis, we kernelized the
genotype, transcriptomic, and phenotype data sets to generate
\(n \times n\) matrices in where \(n\) is the number of individual mice.
Each matrix described the relationships among individuals based on their
genome, transcriptome, or phenome. Each matrix was generated as follows:

\subsubsection{Kernelizing the genome}\label{kernelizing-the-genome}

The kernel matrix of the genome is the overall kinship matrix as
calculated by calc\_kinship() in the R package qtl2 {[}cite{]}. We
further mean-centered this matrix based on DO generation.

\subsubsection{Kernelizing the
transcriptome}\label{kernelizing-the-transcriptome}

Prior to kernelizing the transcriptome, we regressed out the effects of
sex and DO generation. We then mean centered and standardized
transcripts across individuals. The kernel matrix (\(K_t\)) for the
transcriptome of each tissue was calculated as follows:

\begin{equation*}
K_t = \frac{Tr \times Tr^T}{n_{Tx}}
\end{equation*}

where \(Tr\) is a matrix of transcript abundances with individuals in
rows and transcripts in columns, \(Tr^T\) is \(Tr\) transpose, and
\(n_{Tx}\) is the number of transcripts in the matrix. We kernelized
each tissue's transcriptome and then averaged across all tissues to
generate a single transcriptome kernel for all tissues.

\subsubsection{Kernelizing the phenome}\label{kernelizing-the-phenome}

The phenome kernel was constructed the same way as the transcriptome
kernel. We regressed out sex and DO generation and then mean centered
and standardized the phenotypes. We used knn.impute() {[}cite{]} to
impute missing values. We then used the above equation to generate the
phenome kernel replacing the trancript matrix and number of transcripts
with the phenotype matrix and number of phenotypes.

\subsection{High-dimensional mediation
Aanalysis}\label{high-dimensional-mediation-aanalysis}

Matt has text for this section.

\subsection{Calculation of loadings}\label{calculation-of-loadings}

Loadings onto transcripts and traits were calculated in the following
way. calc\_loadings()

\subsection{Enrichment of biological
terms}\label{enrichment-of-biological-terms}

We performed gene set enrichment analysis (GSEA) \cite{pmid16199517}
using the transcript loadings in each tissue as gene weights. GSEA
determines enrichment of pathways based on where the contained genes
appear in a ranked list of genes. If the genes in the pathway are more
concentrated near the top (or the bottom) of the list than expected by
chance, the pathway can be interpreted as being enriched with positively
(negatively) loaded transcripts. We used the R package fgsea
\cite{fgsea} to calculate normalized enrichment scores for all GO terms
and all KEGG pathways.

We downloaded all KEGG \cite{pmid36300620} pathways for
\textit{Mus musculus} using the R package clusterProfiler
\cite{pmid36300620}. We then used fgsea to calculate enrichment scores
in each tissue using the transcript loadings in each tissue as our
ranked list of genes. We reported the normalized enrichment score (NES)
for the 10 pathways with the largest positive NES and the 10 pathways
with the largest negative NES.

We also collected all loadings of transcripts for the genes in each
pathway to compare loadings across all tissues (Supp. Fig. XXX). We used
the R package pathview \cite{pmid23740750} to visualize the loadings
from each tissue in interesting pathways. We scaled the loadings in each
tissue by the maximum absolute value of loadings across all tissues to
compare them across tissues.

We downloaded GO term annotations from Mouse Genome Informatics at the
Jackson Laboratory \cite{pmid33231642}
\url{https://www.informatics.jax.org/downloads/reports/index.html} We
removed gene-annotation pairs labeled with NOT, indicating that these
genes were known not to be involved in these GO terms. We also limited
our search to GO terms with between 80 and 3000 genes. We used the R
package annotate \cite{R_annotate} to identify the ontology of each term
and the R package pRoloc \cite{pmid24413670} to convert between GO terms
and names. As with the KEGG pathways, we used fgsea to calculate a
normalized enrichment score for each GO term and collected loadings for
the transcripts in each term to compare across tissues.

\subsection{TWAS in DO mice}\label{twas-in-do-mice}

We performed a transcriptome-wide analysis (TWAS) {[}cite{]} in the DO
mice to compare to the results of high-dimensional mediation. To perform
TWAS, we fit a linear model to explain variation in each transcript
across the population using the genotype at the nearest marker to the
gene transcription start site (TSS). We used kinship as a random effect
and sex, diet, and DO generation as fixed effects. The predicted
transcript from each of these models was the imputed transcript based
only on the local genotype.

\textbf{equation}

We correlated each imputed transcript with each of the metabolic
phenotypes after adjusting phenotypes for sex, diet, and DO generation.
To calculate significance of these correlations, we performed
permutation testing by shuffling labels of individual mice and
recalculating correlation values. Significant correlations were those
more extreme than any of the permuted values, corresponding to an
empirical \(p\) value of 0. These are transcripts whose locally encoded
expression level was significantly correlated with one of the metabolic
traits. This suggests an association between the genetically encoded
transcript level and the trait, but does not identify a direction of
causation.

\subsection{Literature support for
genes}\label{literature-support-for-genes}

To determine whether each gene among those with large loadings or large
heritability had a supported connection to obesity or diabetes in the
literature, we used the R package easyPubMed {[}cite{]}. We searched for
the terms (``diabetes'' OR ``obesity'') along with the tissue name
(adipose, islet, liver, or muscle), and the gene name. We restricted the
gene name to appear in the title or abstract as some short names
appeared at random in contact information. We checked each gene with
apparent literature support by hand to verify that support, and we
removed spurious associations. For example, FAU is used as an acronym
for fatty acid uptake and CAD is used as an acronym for coronary artery
disease. Both terms co-occur with the terms diabetes and obesity in a
manner independent of the genes \textit{Fau} and \textit{Cad}. Other
genes that co-occurred with diabetes and obesity, but not as a
functional connection were similarly removed. For example, the gene
\textit{Rpl27} is used as a reference gene for quantification of the
expression of other genes, and co-occurrence with diabetes and obesity
is a coincidence. We counted the abstracts associated with diabetes or
obesity and each gene name, and determined that a gene had literature
support when it had at least two abstracts linking it to the terms
diabetes or obesity in the respective tissue.

\subsection{CC-RIX mice}\label{cc-rix-mice}

\subsection{CC-RIX genotypes}\label{cc-rix-genotypes}

We used the most recent common ancestor (MRCA) genotypes for the
Collaborative Cross (CC) mice available on the University of North
Carolina website: \url{http://www.csbio.unc.edu/CCstatus/CCGenomes/}

To generate CC-RIX genotypes, we averaged the haplotype probabilities
for the two parental strains at each locus.

\subsection{Imputation of gene expression in
CC-RIX}\label{imputation-of-gene-expression-in-cc-rix}

To impute gene expression in the CC-RIX, we performed the following
steps for each transcript in each tissue (adipose, liver, and skeletal
muscle): 1. Calculate diploid CC-RIX genotype for all CC-RIX individuals
at the marker nearest the transcription start site of the transcript. 2.
Multiply the genotype probabilities by the eQTL coefficients identified
in the DO population.

To check the accuracy of the imputation, we correlated the each imputed
transcript with the measure transcript. The average Pearson correlation
(r) was close to 0.5 for all three tissues (Supp. Fig. XXXA), and as
expected, the correlation between the imputed transcript and the
measured transcript was highly positively dependent on the local eQTL
LOD score of the transcript (Supp. Fig. XXXB).

\subsection{Prediction of CC-RIX
traits}\label{prediction-of-cc-rix-traits}

We used both measured expression and imputed expression combined with
the results from HDMA in the to predict phenotype in the CC-RIX. The
traits measured in the DO and the CC-RIX were not identical, so we
limited our prediction to body weight, which was measured in both
populations, and was the largest contributor to the phenotype score in
the DO.

For each CC-RIX individual, we multiplied the transcript abundaces
across the transcriptome by the loadings derived from the HDMA in the DO
population (Fig. XXXA). This resulted in a vector with \(n\) elements,
where \(n\) is the number of transcripts in the trancriptome. Each
element was a weigted value that combined the relative abundance of the
transcript with how that abundance affected the phenotype. We averaged
the values in this vector to calculate an overall predicted phenotype
score for the individual CC-RIX animal.

After calculating this predicted phenotype value across all CC-RIX
animals, we correlated the predicted values from each tissue with
measured body weight (Fig. XXXB).

\subsection{Cell type specificity}\label{cell-type-specificity}

We investigated whether the loadings derived from HDMA reflected tissue
composition changes in the DO mice prone to obesity on the high-fat
diet. To do this, we acquired lists of cell-type specific transcripts
from the literature. In adipose tissue, we looked at cell-type specific
transcripts for macrophages, leukocytes, adipocyte progenitors, and
adipocytes as defined in {[}29087381{]}. In pancreatic islets, we looked
at cell-type specific transcripts for alpha cells, beta cells, delta
cells, ductal cells, mast cells, macrophages, acinar cells, stellate
cells, gamma and epsilon cells, and endothelial cells as defined by
{[}36778506{]}. Both studies defined cell-type specific transcripts
based on human cell types. We collected the loadings for each set of
cell-type specific transcripts in the respective tissue and asked
whether the mean loading for the cell type differed significantly from 0
(Figure XXX). A significant positive loading for the cell type would
suggest a genetic predisposition to have a higher proportion of that
cell type in the tissue. To determine whether each mean loading differed
significantly from 0, we performed permutation tests. We randomly
sampled \(n\) genes outside of the cell-type specific, where \(n\) was
the number of genes in the set. We compared the distribution of loading
means over 10,000 random draws to that seen in the observed data. We
used a significance threshold of 0.01.

\subsection{Comparison of transcriptomic signatures to human
transcriptomic
signatures}\label{comparison-of-transcriptomic-signatures-to-human-transcriptomic-signatures}

To compare the transcriptomic signatures identified in the DO mice to
those seen in human patients, we downloaded human gene expression data
from the Gene Expression Omnibus (GEO) {[}cite{]}. We focused on adipose
tissue because this had the strongest relationship to obesity and
insulin resistance in the DO. We downloaded the following human gene
expression data sets:

\begin{itemize}
\item
  Accession number GSE152517 - Performed bulk RNA sequencing on visceral
  adipose tissue resected from seven diabetic and seven non-diabetic
  obese individuals.
\item
  Accession number GSE44000 - Used Agilent-014850 4X44K human whole
  genome platform arrays (GPL6480) to measure gene expression in
  purified adipocytes derived from the subcutaneous adipose tissue of
  seven obese (BMI\textgreater30) and seven lean (BMI\textless25)
  post-menopausal women.
\item
  Accession number GSE205668 - Subcutaneous adipose tissue was resected
  during elective surgery from 35 normal weight, and 26 obese children.
  Gene expression was measured by RNA sequencing with an Illumina HiSeq
  2500.
\item
  Accession number GSE29231 - Visceral adipose biopsies were taken from
  three female patients with type 2 diabetes, and three non-diabetic
  female patients. Expression was measured with Illumina HumanHT-12 v3
  Expression BeadChip arrays.
\end{itemize}

We downloaded each data set from GEO using the R package GEOquery
{[}cite{]}. In each case, we verified that gene expression was log
transdormed, and performed the transformation ourselves if it hadn't
already been done. When covariates such as age and sex were available in
the meta data files we regressed out these variables. We mean cetered
and standardized gene expression across transcripts.

We matched the human gene expression to the mouse gene expression by
pairing orthologs as defined in The Jackson Laboratory's mouse genome
informatics data base (MGI) {[}cite{]}. We multiplied each transcript in
the human data by the adipose tissue loading of its ortholog in the DO
mice. This resulted in a vector of weighted transcript values for each
individual patient based on their own transcriptional profile and the
obesity-related transcriptional signature from the DO analysis. The mean
of this vector for an individual was the prediction of their obesity
status. Higher values indicate a prediction of higher obesity or risk of
metabolic disease based on adipose gene expression. We then compared the
values across groups, either obese and non-obese, or diabetic and
non-diabetic depending on the groups in each study.

\subsection{Connectivity Map Queries}\label{connectivity-map-queries}

The gene expression profiles in the Connectivity Map database are
derived from human cell lines and human primary cultures, and are
indexed by Entrez gene IDs. To query the CMAP database, we identified
the Entrez gene IDs for the human orthologs of the mouse genes expressed
in each tissue. Each CMAP query takes the 150 most up-regulated and the
150 most down-regulated genes in a signature, however, not all human
genes are included in their database. To ensure we had as many genes as
possible in the query, we selected the top and bottom 200 genes with the
most extreme positive and negative loadings respectively. We pasted
these into the CLUE query application available at
\url{https://clue.io/query}.

To identify drugs that reversed the gene signatures in each tissue, we
first filtered to an appropriate cell type: adipocytes (ASC) for the
adipose tissue gene signature, and pancreatic cells (YAPC) for the islet
gene signature. We sorted the results by the correlation score with the
input signature, identifying compounds with large negative correlations
to the input signature as potential drug candidates.

\bibliographystyle{unsrt}
\bibliography{islet.bib}

\end{document}
